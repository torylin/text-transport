\documentclass{article}
\usepackage{graphicx} % Required for inserting images
\usepackage{amsmath}
\usepackage{amssymb}
\usepackage{xcolor}
\usepackage[parfill]{parskip}
\usepackage{bbm}
\usepackage{hyperref}
\usepackage{geometry}
\geometry{margin=1in}
\newcommand{\vl}[1]{\textcolor{red}{[VL: #1]}}
\allowdisplaybreaks


\title{Causal Effect Estimation with Text}
% \author{vlin2 }
\date{}

\begin{document}

\maketitle

\section{Problem setting}

Consider a collection of texts (e.g., documents, sentences, utterances) $\mathcal{X}$, with individual texts $X_i \in \mathcal{X}$.
\begin{itemize}
    \item Let $X_i = \{g(X_i), h(X_i)\}$, where $g(X_i)$ is the text attribute of interest (i.e., the \textit{treatment}) and $h(X_i)$ denotes all other properties of the text $X_i$. 
    \item Let $Y_i(X_i) = Y_i(g(X_i), h(X_i))$ denote the potential \textit{response} or \textit{outcome} of respondent $i$ after reading text $X_i$.
    \item For simplicity, assume
    \begin{itemize}
        \item $g(X_i) \in \{0, 1\}$ $\forall$ $X_i \in \mathcal{X}$.
        \item Each respondent reads only one text. That is, respondent $i$ reads only $X_i$.
    \end{itemize}
\end{itemize}

We are interested in knowing the effect of the text attribute $g(X)$ on the response $Y$. Using the standard causal notation, the \textbf{estimand} $\tau^*$ for the effect of $g(X)$ on $Y$ is given by
\begin{align*}
    \tau^* &= \mathbb{E}_X[Y_i(g(X_i)=1)] - \mathbb{E}_X[Y_i(g(X_i)=0)] \\
    &= \boxed{\sum_{b \in \mathcal{B}} \Big[ \mathbb{E}[Y_i(g(X_i)=1, h(X_i)=b)] - \mathbb{E}[Y_i(g(X_i)=0, h(X_i)=b] \Big]P(h(X_i)=b)}
\end{align*}

Using stochastic notation, we can write $\tau^*$ more simply.
\begin{itemize}
    \item Let $P_{1 g,h}(X)$ denote a distribution such that $g(X)=1$ and $h(X) \sim P^*$, where $P^*$ is some arbitrary probability distribution.
    \item Let $P_{0 g,h}(X)$ denote a distribution such that $g(X)=0$ and $h(X) \sim P^*$.
    \item Let the quantity $\mu(P)$ be defined as follows:
    \begin{align*}
        \mu(P) &= E_{X_i \sim P_{g,h}}[Y_i(X_i)] \\
        &= \frac{1}{N} \sum_{i=1}^N Y_i(X_i)P_{g,h}(X_i)
    \end{align*}
\end{itemize}

This allows us to express $\tau^*$ as
\begin{equation*}
    \boxed{\tau^* = \mu(P_1)-\mu(P_0)}
\end{equation*}

\section{Estimator}

\subsection{Data setting}

Suppose we have some data collected from a randomized trial, in which subjects $i \in [N]$ are shown various texts randomized over $g(X)$ and $h(X)$. These texts are constructed from individual components in a generative way. 
\begin{itemize}
    \item That is, to construct a text where $g(X)=1$, the text may include one of several selected sentences that are chosen to correspond to $g(X)=1$.
    \item Likewise, for various attributes comprising $h(X)=(h_1(X), h_2(X), h_3(X), \dots)$, there may be several candidate texts that correspond to $h_1(X)=1$, several candidate texts that correspond to $h_2(X)=0$, and so on.
\end{itemize}

While we are guaranteed to be able to obtain an unbiased effect estimate for $g(X)$ from this trial (e.g., using the plug-in estimator), the estimate corresponds only to the effect of $g(X)$ \textit{in this specific text setting}. The type of text setting constructed in this trial is fairly artificial, and we believe that the effect may be different depending on the text setting. It is important, for example, to know if this effect still holds in natural text settings.

Therefore, we are interested in knowing the effect of $g(X)$ under text distributions that are different from the initial text distribution. Again, some notation:

\begin{itemize}
    \item Let $P^R_{g,h} = P^R$ denote the \textit{randomization distribution}, or the distribution of texts as constructed in this randomized trial.
    \item Let $P^T_{g,h} = P^T$ denote the \textit{target distribution}, or the text distribution of interest.
    \item Let $\mathbb{P}(A) = \int_A P(X) dx$.
    \item Using the Radon-Nikodym derivative for change of measure (see Section \ref{sec:change_of_measure}), we can write $P^T$ in terms of $P^R$ as
    \begin{equation*}
        P^T(X) = \frac{d\mathbb{P}^T}{d\mathbb{P}^R}(X)P^R(X)
    \end{equation*}
    \item Now, let $(X_i,Y_i(X_i))_{i=1}^n \sim P^R$ correspond to \textit{observed} pairs of features and outcomes sampled from the randomization distribution $P^R$.
\end{itemize}

\subsection{Horvitz-Thompson estimator}

To define our effect estimate in terms of our target distribution $P^T$, we propose using the ratio between $P^T$ and $P^R$ as importance weights in an Horvitz-Thompson estimator. Using the change of measure for $P^T(X)$ we defined previously, this gives us \vl{Do we need to normalize by $\frac{1}{\sum_{i=1}^n \hat{P}^T(X_i)}$ in the first line?}
\begin{align*}
    \hat{\mu}(P) &= \frac{1}{n} \sum_{i=1}^n \frac{\hat{P}^T(X_i)}{P^R(X_i)}Y_i(X_i) \\
    &= \frac{1}{n} \sum_{i=1}^n \frac{\hat{d \mathbb{P}^T}}{d \mathbb{P}^R}(X_i)\frac{P^R(X_i)}{P^R(X_i)} Y_i(X_i) \\
    &=\boxed{\frac{1}{n} \sum_{i=1}^n \frac{\hat{d \mathbb{P}^T}}{d \mathbb{P}^R}(X_i)Y_i(X_i)}
\end{align*}

We can show that $\hat{\mu}(P)$ is an unbiased estimator for $\mu(P^T)$:
\begin{align*}
    \mathbb{E}[\hat{\mu}(P)] &= \mathbb{E}_{X \sim P^R}[\hat{\mu}(P)] \\
    &= \mathbb{E}\left[\frac{1}{n} \sum_{i=1}^n \frac{\hat{d \mathbb{P}^T}}{d \mathbb{P}^R}(X_i)Y_i(X_i)\right] \\
    &=\frac{1}{n}\sum_{i=1}^n \mathbb{E}_{X \sim P^R}\left[\frac{\hat{d \mathbb{P}^T}}{d \mathbb{P}^R}(X_i)Y_i(X_i)\right] \\
    &= \mathbb{E}_{X \sim P^R}\left[\frac{\hat{d \mathbb{P}^T}}{d \mathbb{P}^R}(X_i)Y_i(X_i)\right] \\
    &=\mathbb{E}_{X \sim P^T}[Y_i(X_i)] \\
    &= \mu(P^T)
\end{align*}

Finally, letting $\mathcal{X}$ be the space of all texts, the variance of the estimator is given by:
\begin{align*}
    \text{Var}[\hat{\mu}(P)] &= \text{Var}_X\left[\frac{1}{n} \sum_{i=1}^n \frac{\hat{d \mathbb{P}^T}}{d \mathbb{P}^R}(X_i)Y_i(X_i)\right] \\
    &= \text{Cov}_X\left[\frac{1}{n} \sum_{i=1}^n \frac{\hat{d \mathbb{P}^T}}{d \mathbb{P}^R}(X_i)Y_i(X_i), \frac{1}{n} \sum_{j=1}^n \frac{\hat{d \mathbb{P}^T}}{d \mathbb{P}^R}(X_j)Y_j(X_j)\right] \\
    &= \frac{1}{n^2}\sum_{i=1}^n \sum_{j=1}^n \text{Cov}_X\left[\frac{\hat{d \mathbb{P}^T}}{d \mathbb{P}^R}(X_i)Y_i(X_i), \frac{\hat{d \mathbb{P}^T}}{d \mathbb{P}^R}(X_j)Y_j(X_j)\right] \\
    &= \frac{1}{n^2}\sum_{i=1}^n \sum_{j=1}^n \text{Cov}_X\left[\sum_{x \in \mathcal{X}}\frac{\hat{d \mathbb{P}^T}}{d \mathbb{P}^R}(X_i)Y_i(X_i)\mathbbm{1}\{X_i = x\}, \sum_{x' \in \mathcal{X}}\frac{\hat{d \mathbb{P}^T}}{d \mathbb{P}^R}(X_j)Y_j(X_j)\mathbbm{1}\{X_j=x'\}\right] \\
    &= \frac{1}{n^2}\sum_{i=1}^n \sum_{j=1}^n \text{Cov}_X\left[\int_x\frac{\hat{d \mathbb{P}^T}}{d \mathbb{P}^R}(X_i)Y_i(X_i)\mathbbm{1}\{X_i = x\}d\mathbb{P}^R(x), \int_{x'}\frac{\hat{d \mathbb{P}^T}}{d \mathbb{P}^R}(X_j)Y_j(X_j)\mathbbm{1}\{X_j=x'\}d\mathbb{P}^R(x')\right] \\
    &= \frac{1}{n^2} \sum_{i=1}^n \sum_{j=1}^n \int_x \int_{x'} \frac{d \mathbb{P}^T}{d \mathbb{P}^R}(X_i)\frac{d \mathbb{P}^T}{d \mathbb{P}^R}(X_j)Y_i(X_i)Y_j(X_j) \text{Cov})_X[\mathbbm{1}\{X_i=x\}, \mathbbm{1}\{X_i=x'\}]d\mathbb{P}^R(x)d\mathbb{P}^R(x') \\
    &= \frac{1}{n^2} \sum_{i,j \in [n]} \iint_{x, x'} \frac{d \mathbb{P}^T}{d \mathbb{P}^R}(X_i)\frac{d \mathbb{P}^T}{d \mathbb{P}^R}(X_j)Y_i(X_i)Y_j(X_j) \text{Cov}_X[\mathbbm{1}\{X_i=x\}, \mathbbm{1}\{X_i=x'\}]d\mathbb{P}^R(x)d\mathbb{P}^R(x') \\
    &= \frac{1}{n^2} \sum_{i,j \in [n]} \iint_{x, x'} \frac{d \mathbb{P}^T}{d \mathbb{P}^R}(X_i)\frac{d \mathbb{P}^T}{d \mathbb{P}^R}(X_j)Y_i(X_i)Y_j(X_j) (P^R(X_i,X_j) - P^R(X_i)P^R(X_j))d\mathbb{P}^R(x)d\mathbb{P}^R(x')
\end{align*}

where the last equality follows from
\begin{align*}
    \text{Cov}[\mathbbm{1}\{X_i=x\},\mathbbm{1}\{X_j=x'\}] &=\mathbb{E}_X[\mathbbm{1}\{X_i=x\}\mathbbm{1}\{X_j=x'\}] - \mathbb{E}_X[\mathbbm{1}\{X_i=x\}]\mathbb{E}_X[\mathbbm{1}\{X_j=x'\}] \\
    &= P^R(X_i,X_j) - P^R(X_i)P^R(X_j)
\end{align*}

With the central limit theorem (CLT), we establish asymptotic normality:
\begin{equation*}
    \frac{\hat{\mu}(P) - \mu(P)}{\sqrt{\text{Var}[\hat{\mu}(P)]}}\rightarrow N(0,1)
\end{equation*}

which we can use to estimate confidence intervals using the following unbiased estimate for the variance.
\begin{equation*}
    \widehat{\text{Var}}[\hat{\mu}(P)] = \frac{1}{n^2} \sum_{i,j \in [n]} \frac{\hat{d \mathbb{P}^T}}{d \mathbb{P}^R}(X_i)\frac{\hat{d \mathbb{P}^T}}{d \mathbb{P}^R}(X_j)Y_i(X_i)\frac{P^R(X_i,X_j) - P^R(X_i)P^R(X_j)}{P^R(X_i,X_j)}Y_j(X_j)
\end{equation*}

\section{Empirical estimation}
\subsection{Estimating $\hat{P}^T$ from the data}

Our estimator has one primary nuisance parameter, $\frac{\hat{d\mathbb{P}^T}(X)}{d\mathbb{P}^R(X)}$ (or $\hat{P}^T(X)$, depending on the version of the estimator we want to use). We propose several approaches for estimating this quantity.

\subsubsection{Classification}
\label{sec:classification}

We notice that we can rewrite $\frac{d\mathbb{P}^T}{d\mathbb{P}^R}$. Let $C$ denote the distribution (or corpus) from which a text is drawn, where $C=T$ denotes that it is drawn from $P^T$ and $C=R$ denotes that it is drawn from $P^R$. Then
\begin{equation*}
    \begin{split}
        &\frac{d\mathbb{P}^T}{d\mathbb{P}^R}(X) = \frac{P(C=T|X)}{P(C=T)}\frac{P(C=R)}{P(C=R|X)}\\
        \Rightarrow &\frac{\hat{d\mathbb{P}^T}}{d\mathbb{P}^R}(X) = \frac{\hat{P}(C=T|X)}{\hat{P}(C=T)}\frac{\hat{P}(C=R)}{\hat{P}(C=R|X)}
    \end{split}
\end{equation*}

where estimation of $\hat{P}(C=T|X)$ and $\hat{P}(C=R|X)$ is straightforward---we can train a binary classifier $M_\theta: \mathcal{X} \rightarrow \{0,1\}$ to predict if a text $X$ came from $T$ or $R$---and $\hat{P}(C=R)$ and $\hat{P}(C=T)$ are estimated from their sample proportions.

We have several options for our binary classifier:
\begin{enumerate}
    \item A model that takes ``interpretable'' language features as input (e.g., bag-of-words, lexicon, SenteCon). We train a ``simple'' classifier over these features (e.g., logistic regression, SVM).
    \item A deep language model that takes raw text as input. We can either use a pre-trained model, or we can fine-tune the pre-trained model using masked language modeling (MLM) on text examples drawn from $P^T$.
\end{enumerate}

\subsubsection{Deep language tasks}

Deep language models like transformers are capable of directly computing the probability of a sentence (see Section \ref{sec:huggingface_sentence_probs}), even if the model has never seen the sentence before. This probability is based on the text distribution used to train the model. Since the state-of-the-art language models are pre-trained on extremely large corpora that approximate the entirety of the English language, we must further fine-tune before we can use them to compute sentence probabilities for $P^T$.

Several tasks that may be useful for encouraging a language model $M_\theta$ to learn $P^T$:
\begin{itemize}
    \item MLM: Mentioned in Section \ref{sec:classification}. We draw text examples from $P^T$. For each sentence, certain words are masked, and $M_\theta$ learns to predict them.
    \item Causal language modeling (CLM)/text generation: Again, we drawn text samples from $P^T$. For each sentence, starting from the first word, $M_\theta$ learns to predict the next word.
    \item Style transfer: Given a text example drawn from $P^R$, $M_\theta$ reproduces that same text example ``in the style of'' $P^T$. This task requires $M_\theta$ to be a sequence-to-sequence language model, like \href{https://arxiv.org/pdf/1910.13461.pdf}{BART}, and for best results, parallel text examples from $P^R$ and $P^T$ should be available.
\end{itemize}

For a practical example of MLM and CLM, see \href{https://huggingface.co/docs/transformers/tasks/language_modeling}{this example from HuggingFace}. For some ideas on how to fine-tune BART on a style transfer task, see \href{https://blog.fastforwardlabs.com/2022/05/05/neutralizing-subjectivity-bias-with-huggingface-transformers.html}{this blog post} and \href{https://arxiv.org/pdf/2105.06947.pdf}{this paper}.

\subsection{Estimating $\hat{\mu}(P)$ from the data}

We estimate the overall quantity of interest $\hat{\tau}$ through the following procedure. Suppose we have samples $(X_i,Y_i)_{i=1}^n \sim P^R$. In our experiments, these samples are from this \href{https://onlinelibrary.wiley.com/doi/full/10.1111/ajps.12649}{2021 Fong and Grimmer paper}. The authors conduct a randomized study on the effects of a text attribute $g(X)$ on an outcome $Y$ in a constructed text setting, which we consider to be distributed $P^R$.
\begin{enumerate}
    \item Sample examples from $P^T$. Train $M_\theta$ on these examples to be able to obtain $\frac{\hat{d \mathbb{P}^T}}{d \mathbb{P}^R}(X_i)$ or $\hat{P}^T(X_i)$ $\forall$ $i \in [n]$.
    \begin{itemize}
        \item In our experiments, these examples from $P^T$ constitute 2-3 sentence chunks drawn from the U.S. Congressional speeches on the Hong Kong protests. We consider these to be the ``natural language counterparts'' of the examples from $P^R$.
    \end{itemize}
    \item (If using $P^T$) $P^R(X_i)$ should be known from the design of the randomized study. In the worst case, it can be computed empirically from sample proportions.
    \item Split $(X_i,Y_i)$ according to samples where $g(X_i)=1$ and where $g(X_i)=0$. Call these data splits $D_1$ and $D_0$, respectively.
    \item Compute
    \begin{equation*}
    \begin{split}
        \hat{\tau} &= \hat{\mu}(P_1) - \hat{\mu}(P_0) \\
        &= \frac{1}{|D_1|} \sum_{j \in D_1} \frac{\hat{d \mathbb{P}^T}}{d \mathbb{P}^R}(X_j)Y_j - \frac{1}{|D_0|} \sum_{k \in D_0} \frac{\hat{d \mathbb{P}^T}}{d \mathbb{P}^R}(X_k)Y_k \\
        &= \frac{1}{|D_1|} \sum_{j \in D_1} \frac{\hat{P}(C=T|X_j)}{\hat{P}(C=T)}\frac{\hat{P}(C=R)}{\hat{P}(C=R|X_j)}
    \end{split}
    \end{equation*}
    or alternatively
    \begin{equation*}
    \begin{split}
        \hat{\tau} &= \hat{\mu}(P_1) - \hat{\mu}(P_0) \\
        &= \frac{1}{|D_1|} \sum_{j \in D_1} \frac{\hat{P}^T(X_j)}{P^R(X_j)\sum_{i=1}^n \hat{P}^T(X_i)}Y_j - \frac{1}{|D_0|} \sum_{k \in D_0} \frac{\hat{P}^T(X_k)}{P^R(X_k)\sum_{i=1}^n \hat{P}^T(X_i)}Y_k
    \end{split}
    \end{equation*}
    \vl{Is the normalization just to ensure that $\hat{P}^T$ sums to 1 over $X_i \sim P^R$?}
\end{enumerate}

\section{Extra notes}

\vl{These two subsections were generated using ChatGPT and are mostly for my own reference.}

\subsection{Change of measure with Radon-Nikodym derivatives}
\label{sec:change_of_measure}

The Radon-Nikodym derivative can be used to express one probability density function in terms of another probability density function, when the two densities are related by a change of measure. Specifically, if we have two probability measures defined on the same sample space, with one measure $\mathbb{P}$ absolutely continuous with respect to another measure $\mathbb{Q}$, then there exists a Radon-Nikodym derivative $Z$ such that:

$$\mathbb{P}(A) = \int_A Z d\mathbb{Q}$$

for any event $A$ in the sample space. Intuitively, this means that we can define the probability of any event under the measure $\mathbb{P}$ in terms of the probability of the same event under the measure $\mathbb{Q}$, by weighting the probabilities by a factor given by the Radon-Nikodym derivative.

Now, suppose we have two probability density functions $p(x)$ and $q(x)$ defined on some real-valued random variable $X$, with $q(x)>0$ for all $x$. We can interpret $q(x)$ as the "reference" density, and we want to express $p(x)$ in terms of $q(x)$ by a change of measure. To do this, we can define a new probability measure $\mathbb{P}$ as:
$$\mathbb{P}(A) = \int_A \frac{p(x)}{q(x)} q(x) dx = \int_A p(x) dx$$

for any event $A$ in the sample space. This means that we are weighting the probability of each event in proportion to the ratio $p(x)/q(x)$, which is a function of $x$. We can show that $\mathbb{P}$ is absolutely continuous with respect to the measure defined by $q(x)$, and therefore there exists a Radon-Nikodym derivative $Z(x)$ such that:
$$\frac{d\mathbb{P}}{d\mathbb{Q}}(x) = Z(x) = \frac{p(x)}{q(x)}$$

This means that we can express $p(x)$ in terms of $q(x)$ and the Radon-Nikodym derivative $Z(x)$, as:
$$p(x) = Z(x) q(x)$$

This is a general formula for expressing one probability density function in terms of another probability density function by a change of measure.

\subsection{Sentence probabilities from deep language models}
\label{sec:huggingface_sentence_probs}

To get the probability of a sentence from a HuggingFace transformers model, you first need to convert the sentence into a format that the model can understand. Typically, this involves tokenizing the sentence into a sequence of subwords, which is done using the tokenizer that corresponds to the pre-trained model you are using.

Once you have tokenized the sentence, you can pass it to the model to get the probability distribution over the possible output tokens. The probability of the sentence can be computed as the product of the probabilities of each individual token in the sequence.

Here's an example of how to get the probability of a sentence using the HuggingFace Transformers library in Python:

\begin{verbatim}
from transformers import pipeline, AutoTokenizer, AutoModelForMaskedLM
import torch

# Load the pre-trained model and tokenizer
model_name = "distilbert-base-uncased"
tokenizer = AutoTokenizer.from_pretrained(model_name)
model = AutoModelForMaskedLM.from_pretrained(model_name)

# Define the input sentence
input_sentence = "The cat sat on the mat."

# Tokenize the input sentence
tokens = tokenizer.encode(input_sentence, return_tensors="pt")

# Get the probability distribution over the tokens
outputs = model(tokens)
predictions = torch.nn.functional.softmax(outputs.logits, dim=-1)

# Compute the probability of the sentence
probability = 1.0
for i in range(1, len(tokens[0])):
    probability *= predictions[0, i, tokens[0, i]].item()

print("The probability of the sentence is:", probability)
\end{verbatim}

This code uses the AutoTokenizer and AutoModelForMaskedLM classes from the Transformers library to load a pre-trained model and tokenizer. It then tokenizes the input sentence using the tokenizer, passes the resulting tokens to the model, and computes the probability of the sentence by multiplying together the probabilities of the individual tokens.

% In probability theory, a change of measure refers to the process of transforming a probability distribution from one measure to another. This can be useful for a variety of purposes, such as simplifying calculations, deriving new distributions, or solving certain statistical problems.

% A probability measure is a mathematical function that assigns a probability to each event in a sample space. For example, in the case of a continuous probability distribution, a probability measure might be defined as a function that assigns probabilities to intervals on the real line. Different probability measures can be defined on the same sample space, and they can be related to each other through a change of measure.

% One common type of change of measure is called a Radon-Nikodym derivative, which is a way of transforming a probability measure into a new measure that is equivalent in some sense. Specifically, if $\mathbb{P}$ and $\mathbb{Q}$ are two probability measures defined on the same sample space, with $\mathbb{Q}$ absolutely continuous with respect to $\mathbb{P}$, then there exists a Radon-Nikodym derivative $Z$ such that:

% $$\mathbb{Q}(A) = \int_A Z d\mathbb{P}$$

% for any event $A$ in the sample space. Intuitively, this means that we can define a new probability measure $\mathbb{Q}$ in terms of the original measure $\mathbb{P}$, by weighting the probabilities of each event by a factor given by the Radon-Nikodym derivative.

% Change of measure can be used in various areas of probability theory, such as in stochastic calculus, finance, and statistics. For example, in financial mathematics, change of measure is used to transform a probability distribution from a historical measure (which reflects past market behavior) to a risk-neutral measure (which reflects the market's expectation of future behavior), which can be used to price financial derivatives. In statistics, change of measure can be used to transform the distribution of a statistic, for example, to derive the distribution of a test statistic under a null hypothesis.

% The Horvitz-Thompson estimator is a well-known method for estimating population totals or means from survey data, when the sampling design is known. The basic idea is to weight the observed values by the inverse of the probability of selection of the sampling units, which gives an unbiased estimate of the total or mean of the population.

% In some cases, the Horvitz-Thompson estimator can be improved by using a change of measure based on the Radon-Nikodym derivative. Specifically, suppose we have a sampling design where each unit has a known inclusion probability $p_i$, and we want to estimate the population mean of a variable $Y$ based on a sample of size $n$. The Horvitz-Thompson estimator for the population mean is:

% $$\widehat{\mu}_{HT} = \frac{1}{n} \sum_{i \in s} \frac{Y_i}{p_i}$$

% where $s$ is the set of sampled units. If we define a new probability measure $\mathbb{Q}$ as:

% $$\mathbb{Q}(A) = \sum_{i \in A} \frac{1}{np_i}$$

% for any event $A$ in the sample space, then we can show that $\mathbb{Q}$ is absolutely continuous with respect to the sampling design probability measure $\mathbb{P}$. This means that we can use the Radon-Nikodym derivative $Z$ to define a new set of weights $w_i$ for the sampled units, as:

% $$w_i = \frac{Z_i}{np_i}$$

% where $Z_i$ is the value of the Radon-Nikodym derivative for unit $i$. Intuitively, the new weights adjust the original Horvitz-Thompson weights to reflect the change of measure, and they can be used to define a new estimator for the population mean:

% $$\widehat{\mu}_{QN} = \frac{\sum_{i \in s} w_i Y_i}{\sum_{i \in s} w_i}$$

% This estimator is known as the Quenouille-Tukey estimator, and it has been shown to have better properties than the Horvitz-Thompson estimator in certain situations, such as when the inclusion probabilities are highly variable. However, the Radon-Nikodym derivative and the new weights are not always easy to compute in practice, and the estimator may not be robust to model misspecification or outliers. Therefore, the use of change of measure with the Horvitz-Thompson estimator should be carefully considered and justified based on the specific properties of the sampling design and the underlying population distribution.


% Suppose we have a randomized controlled trial (RCT) with two treatment groups (treatment and control) and a binary outcome variable $Y$, indicating whether a patient recovers or not. Our estimand of interest is the average treatment effect (ATE) on the treated, which is the difference in the probabilities of recovery between the treatment group and the control group. Mathematically, the ATE on the treated is defined as:

% $$\text{ATE} = \mathbb{E}[Y_i^{\text{T}} - Y_i^{\text{C}} | \text{T}_i = 1]$$

% where $Y_i^{\text{T}}$ and $Y_i^{\text{C}}$ are the potential outcomes for patient $i$ under treatment and control, respectively, and $\text{T}_i$ is the treatment assignment indicator (equal to 1 for patients in the treatment group and 0 for patients in the control group). The expected value is taken over the population of patients who receive treatment, i.e., those for whom $\text{T}_i = 1$.

% One common estimator for the ATE on the treated is the difference in the sample means of the outcome variable between the treatment and control groups, restricted to the treated patients. Mathematically, the estimator is given by:

% $$\widehat{\text{ATE}} = \frac{1}{n_{\text{T}}} \sum_{i=1}^{n_{\text{T}}} Y_i^{\text{T}} - \frac{1}{n_{\text{C}}} \sum_{i=1}^{n_{\text{C}}} Y_i^{\text{C}}$$

% where $n_{\text{T}}$ and $n_{\text{C}}$ are the sample sizes of the treatment and control groups, respectively, and the sums are taken over the treated patients (those for whom $\text{T}_i = 1$) in each group.

% In this example, the estimand is the ATE on the treated, which is a theoretical quantity of interest that we want to estimate from the RCT data. The estimator is the difference in the sample means of the outcome variable between the treatment and control groups, restricted to the treated patients, which is a statistic computed from the observed data.

\end{document}
 